\documentclass[a4paper]{article}

\usepackage[latin1]{inputenc}
\usepackage[T1]{fontenc}
\usepackage[francais]{babel}
\usepackage[right=3cm, bottom=4cm]{geometry}
\usepackage{lmodern}
\usepackage{pdfpages}
\usepackage{titlesec}
\usepackage{array}
\usepackage{color}
\usepackage{alltt}
\usepackage{verbatim}
\usepackage{ulem}
\usepackage{latexsym}
\usepackage{amsmath, amsfonts, amssymb}

\title{
\textsc{Distributed Systems\\
\LARGE An Automated And Personal Touristic Tour Application Using Google Maps API }
}

\author
{
	Quentin {\sc Augrain}\\
    Florent {\sc Mallard}\\
}
\newcommand{\question}[2]{\paragraph{#1} \paragraph{}#2}
\date{\today}

\begin{document}
\maketitle

\begin{abstract}
	Tourism is one of the most lucrative business today. Many companies organize city tours for a living, and can provide pieces of advice to tourists. With Distributed Systems, and with the help of the Internet, you can find by yourself the most famous places of the city you are currently in. From there, you just have to find their location and try to find the better way to go see all of them. Our idea is to provide the user an application using Google Maps API, that allows him to calculate the better way to go through all the monuments he wants to visit. The application will ask the location, and then register the list of the places the user wants to go to. After that, it will optimize your way through the city and organize your city tour. It will display the distance and traveling time between each visit, and at the end tell you the total distance you will have to cover. The application will eventually display a list of recommended places to see, in order to give a starting point in case the user doesn't know the city at all.
\end{abstract}

\end{document}